\documentclass[12pt]{amsart}

\addtolength{\hoffset}{-2.25cm}
\addtolength{\textwidth}{4.5cm}
\addtolength{\voffset}{-2.5cm}
\addtolength{\textheight}{5cm}
\setlength{\parskip}{0pt}
\setlength{\parindent}{15pt}

\usepackage{amsthm}
\usepackage{amsmath}
\usepackage{amssymb}
\usepackage[spanish]{babel}
\usepackage[colorlinks = true, linkcolor = black, citecolor = black, final]{hyperref}

\usepackage{listings}
\lstset{
	language=Python,
	backgroundcolor=\color{gray!25},
	basicstyle=\small\ttfamily,
	breaklines=true
}

\usepackage{graphicx}
\usepackage{multicol}
\usepackage{ marvosym }
\usepackage{wasysym}
\usepackage{tikz}
\usetikzlibrary{patterns}

\newcommand{\ds}{\displaystyle}
\newcommand{\horrule}[1]{\rule{\linewidth}{#1}}

\setlength{\parindent}{0in}

\pagestyle{empty}
\begin{document}
	\hrule
	\smallskip
	\begin{center}
		{\scshape {\large Redes de Computadoras} \\
			Curso 2025-2026} \\ \smallskip
		\textbf{Clase Práctica \# 7} \\
		{\small \textbf{Tema:} Protocolos de capa de aplicación: DNS}
	\end{center}
	\vspace{-8px}
	\rule{\linewidth}{2pt}
	
	{\scshape Facultad de Matemática y Computación}  \hfill {\scshape Universidad de La Habana}
	
	\bigskip\bigskip
	
	
	
	\vspace{1cm}
	
	\begin{center}
		\textit{La única forma de hacer un gran trabajo es amar lo que haces.}
	\end{center}
	
	\begin{flushright}
		- Steve Jobs
	\end{flushright}
	
	\vspace{1cm}
	
	% Ejercicios
	\begin{enumerate}
		
		\thispagestyle{empty}
		
		%------------------------------------------------------------%
		\item \textbf{\textit{Fase 1: Exploración rápida}}
		
		\medskip
		Comando principal:
		\begin{lstlisting}
dig +trace www.example.com
		\end{lstlisting}
		\medskip \medskip
		
		\noindent \textbf{Conceptos clave::}

		\begin{itemize}
			
		\item Raíz: conjunto de servidores que conoce ubicación de TLD; inicio del árbol.

		\item TLD: delega hacia servidores autoritativos del dominio.

		\item Autoritativo final: posee los registros definitivos del nombre consultado.

		\item Secciones: ANSWER (respuesta directa), AUTHORITY (quién tiene autoridad), ADDITIONAL (datos extra como IP de NS) para facilitar siguiente salto.

		\item RD/RA: diferencia entre petición de recursión y capacidad de ofrecerla.

		\end{itemize}

		\bigskip\bigskip
		%------------------------------------------------------------%

		%------------------------------------------------------------%
		\item \textbf{\textit{ Fase 2: Captura y disección (Wireshark)}}
		
		\medskip
		Pasos:
		
		\medskip \medskip

		\begin{enumerate}
		\item Abrir Wireshark y seleccionar interfaz con salida a Internet.
		
		\item Iniciar captura; aplicar filtro de visualización: dns (o udp.port==53 si se quiere ver sólo UDP clásico).

		\item Ejecutar en terminal:
		\begin{lstlisting}
dig A example.com
dig AAAA example.com
dig TXT example.com
		\end{lstlisting}
		
		\item Detener captura tras 3–4 segundos.

		\item En un paquete de respuesta: expandir "Domain Name System (response)" y ubicar: Transaction ID, Flags (ver bits QR, AA, RD, RA), RCODE, Section Answers.

		\item Repetir con otra consulta y comparar diferencias (por ejemplo presencia/ausencia de AAAA).
		\end{enumerate}

		\medskip \medskip
		Marcar manualmente (notas): ID, QR, AA, RD, RA, RCODE de 3 respuestas distintas. Añadir columna en Wireshark (Click derecho sobre campo → "Apply as Column") para Transaction ID y RCODE facilita comparación.
		\medskip \medskip
		\noindent \textbf{Conceptos clave:}
		\begin{itemize}
			
		\item ID: correlaciona consulta/respuesta.

		\item QR: tipo de mensaje (0 consulta / 1 respuesta).

		\item AA: confirma autoridad sobre la zona de la respuesta.

		\item RD/RA: negociación de recursión.

		\item RCODE: estado de la operación (NOERROR, NXDOMAIN). Permite rápida interpretación del resultado.

		\item Filtro dns : muestra sólo tráfico relevante, reduciendo ruido y carga cognitiva.

		\end{itemize}
		
		\bigskip\bigskip
		%------------------------------------------------------------%

		%------------------------------------------------------------%
		\item \textbf{\textit{Fase 3: Servidor autoritativo mínimo}}
		
		\medskip
		Docker rápido (ejemplo Bind9 simplificado):
		\begin{lstlisting}
docker run -d --name dns --hostname ns.grupoX.local -p 5353:53/udp -v $(pwd)/zone:/etc/bind
ubuntu/bind9
		\end{lstlisting}
		\medskip \medskip
	
		Archivo de zona (ejemplo grupoX.local.zone ): incluir SOA + NS + registros: A (www), AAAA (www6), CNAME (alias -> www), MX (10 mail), TXT ("v=spf1 -all"), SRV (demo.tcp). Consultar:
		\begin{lstlisting}
dig @127.0.0.1 -p 5353 grupoX.local SOA
dig @127.0.0.1 -p 5353 www.grupoX.local A
		\end{lstlisting}
		
		\medskip \medskip
		Repetir tras unos segundos para ver efecto de TTL en caché local.
		
		\medskip \medskip
		\noindent \textbf{Conceptos clave:}
		\begin{itemize}
			
		\item Archivo de zona: declaración de datos autoritativos.

		\item TTL: límite de vigencia en cachés; su decremento muestra temporalidad

		\item Diferencia autoridad vs caché: autoridad siempre devuelve valor completo; caché puede devolver mientras TTL > 0.

		\item Tipos listados (A, AAAA, CNAME, MX, TXT, SRV) contextualizan diversidad sin exigir configuración completa.

		\end{itemize}

		\bigskip\bigskip
		%------------------------------------------------------------%
		
		%------------------------------------------------------------%
		\item \textbf{\textit{Fase 4: Cliente manual Python (mínimo)}}
		
		\medskip
		Objetivo: Construir paquete DNS (header 12 bytes + question) y parsear respuestas A/AAAA/TXT.

		\medskip \medskip
		\noindent \textbf{Conceptos clave:}
		\begin{enumerate}
			
		\item Empaquetar ID aleatorio, flags = 0x0100 (RD=1), QDCOUNT=1,

		\item Codificar nombre en labels, QTYPE/QCLASS,

		\item Enviar por UDP a resolver público (ej. 1.1.1.1),

		\item Parsear header y contar answers.

		\end{enumerate}

		\medskip \medskip
		\noindent \textbf{Conceptos clave:}
		\begin{itemize}
			
		\item Header (12 bytes): estructura fija que identifica y clasifica la pregunta.

		\item Labels codificados: longitud + texto + terminador 0x00 evita ambigüedad.

		\item Flags mínimos: usar sólo RD simplifica; otros bits se exploran posteriormente.

		\item QTYPE/QCLASS: definen qué se busca y el contexto (Internet).

		\item Parseo parcial: suficiente para entender respuesta sin construir parser completo de RRs.

		\end{itemize}

		\bigskip\bigskip
		%------------------------------------------------------------%

	\end{enumerate}	

\end{document}
