\documentclass[12pt]{amsart}

\addtolength{\hoffset}{-2.25cm}
\addtolength{\textwidth}{4.5cm}
\addtolength{\voffset}{-2.5cm}
\addtolength{\textheight}{5cm}
\setlength{\parskip}{0pt}
\setlength{\parindent}{15pt}

\usepackage{amsthm}
\usepackage{amsmath}
\usepackage{amssymb}
\usepackage[spanish]{babel}
\usepackage[colorlinks = true, linkcolor = black, citecolor = black, final]{hyperref}

\usepackage{listings}
\lstset{
	language=Python,
	backgroundcolor=\color{gray!25},
	basicstyle=\small\ttfamily,
	breaklines=true
}

\usepackage{graphicx}
\usepackage{multicol}
\usepackage{ marvosym }
\usepackage{wasysym}
\usepackage{tikz}
\usetikzlibrary{patterns}

\newcommand{\ds}{\displaystyle}
\newcommand{\horrule}[1]{\rule{\linewidth}{#1}}

\setlength{\parindent}{0in}

\pagestyle{empty}
\begin{document}
	\hrule
	\smallskip
	\begin{center}
		{\scshape {\large Redes de Computadoras} \\
			Curso 2025-2026} \\ \smallskip
		\textbf{Clase Práctica \# 9} \\
		{\small \textbf{Tema:} Seguridad}
	\end{center}
	\vspace{-8px}
	\rule{\linewidth}{2pt}
	
	{\scshape Facultad de Matemática y Computación}  \hfill {\scshape Universidad de La Habana}
	
	\bigskip\bigskip
	
	
	
	\vspace{1cm}
	
	\begin{center}
		\textit{La única forma de hacer un gran trabajo es amar lo que haces.}
	\end{center}
	
	\begin{flushright}
		- Steve Jobs
	\end{flushright}
	
	\vspace{1cm}
	
	% Ejercicios
	\begin{enumerate}
		
		\thispagestyle{empty}
		
		%------------------------------------------------------------%
		\item \textbf{\textit{¿Qué es un Firewall?}}
		
		\medskip
		Un firewall es un componente de seguridad que \textbf{aplica políticas de control de tráfico} entre redes o entre clientes y servicios. Decide permitir, bloquear o limitar conexiones según reglas.
		\medskip \medskip
		
		\noindent \textbf{Tipos por capa:}
		\begin{itemize}
		\item \textbf{Capa de Red / Capa de Transporte}: Basado en IP (origen/destino), puertos y estado de conexión (TCP/UDP). Ej.: \textbf{iptables}/\textbf{nftables} aplican filtros a estas capas.
		\medskip

		\item \textbf{Capa de Aplicación}: Inspecciona contenido y semántica del protocolo (HTTP headers, métodos, rutas, payload). Ej.: Nginx y WAFs que operan sobre HTTP/HTTPS.
		\end{itemize}

		\bigskip\bigskip
		%------------------------------------------------------------%

		%------------------------------------------------------------%
		\item \textbf{\textit{ ¿Contra qué servidor probamos?}}
		
		\medskip
		En estos ejercicios usamos \textbf{Nginx local} como servidor HTTP/HTTPS de prueba. Todas las solicitudes \textbf{curl} van contra http://localhost (puerto 80) o https://localhost (puerto 443) según el bloque server configurado.
		
		\bigskip\bigskip
		%------------------------------------------------------------%

		%------------------------------------------------------------%
		\item \textbf{\textit{ Criptografía Simétrica vs Asimétrica}}
		
		\medskip
		\begin{itemize}
		\item Simétrica: Un mismo secreto k cifra y descifra. Rápida (AES-GCM/ChaCha20-Poly1305). En TLS se usa para el canal de datos tras el handshake.
		\medskip
		
		\item Asimétrica: Par de claves (privada, pública). Se usa para intercambio de claves y firmas (RSA/ECDSA/Ed25519). Más costosa, ideal para autenticación y establecimiento de secretos.
		\medskip
		
		\item TLS/HTTPS: Usa asimetría para autenticación del servidor e intercambio de claves; luego simetría para el tráfico.
		\end{itemize}.
		
		\bigskip\bigskip
		%------------------------------------------------------------%

		%------------------------------------------------------------%
		\item \textbf{\textit{¿Qué es TLS y el Handshake?}}
		
		\medskip
		\begin{itemize}
		\item \textbf{TLS (Transport Layer Security)}: Protocolo que proporciona \textbf{confidencialidad}, \textbf{integridad} y, generalmente, \textbf{autenticación} para comunicaciones sobre TCP (HTTPS es HTTP sobre TLS).

		\medskip \medskip		

		\item Handshake TLS:
		\begin{enumerate}

		\item Cliente envía \textbf{ClientHello}
		\medskip

		\item Servidor responde \textbf{ServerHello}, y envía su \textbf{certificado} (clave pública).
		\medskip

		\item Cliente valida la cadena del certificado y el hostname; se negocia un secreto compartido mediante Diffie-Hellman/ECDHE.
		\medskip

		\item Se derivan claves simétricas de sesión y se empieza a cifrar datos.
		\end{enumerate}
		\end{itemize}.
		
		\bigskip\bigskip
		%------------------------------------------------------------%

		%------------------------------------------------------------%
		\item \textbf{\textit{¿Qué es OpenSSL?}}
		
		\medskip
		\textbf{OpenSSL)}: Toolkit criptográfico y librería ampliamente usada. Proporciona comandos para \textbf{conectar vía TLS}, \textbf{inspeccionar certificados} (x509), \textbf{generar claves}, y probar suites.
		
		\medskip \medskip
		En esta clase usamos `openssl s\_client` (cliente TLS de prueba) y `openssl x509` (inspección de certificados X.509).
		
		\bigskip\bigskip
		%------------------------------------------------------------%

		%------------------------------------------------------------%
		\item \textbf{\textit{Ejercicios}}
		
		\begin{itemize}

		\item \noindent \textbf{Parte A: Nginx como Firewall de Capa de Aplicación}

		\medskip
		Prerequisitos:
		\begin{itemize}
		\item Tener privilegios para instalar paquetes y editar `/etc/nginx/`.
		\medskip
		\item Puertos disponibles: `80` (HTTP) y `443` (HTTPS) si pruebas con TLS.
		\end{itemize}
		
		\medskip \medskip
		En esta clase usamos `openssl s\_client` (cliente TLS de prueba) y `openssl x509` (inspección de certificados X.509).
		
		\medskip \medskip
		\begin{enumerate}
		\item Instalación y verificación
		\begin{lstlisting}
sudo apt update
sudo apt install -y nginx curl
sudo systemctl enable nginx
sudo systemctl start nginx
sudo systemctl status nginx --no-pager
		\end{lstlisting}
		\medskip \medskip

		\item Configurar un sitio simple con reglas básicas

		Por qué: Un backend mínimo nos permite aislar comportamientos y validar políticas sin ruido.

		Edita `/etc/nginx/sites-available/app.conf` con un server básico y políticas L7:

		\begin{lstlisting}
server {
	listen 80 default_server;
	server_name _;

	# Regla 1: Bloquear metodos peligrosos en /public
	location /public {
		if ($request_method ~* "(POST|PUT|DELETE)") { return 405; }
		return 200 "public ok";
	}

	# Regla 2: Bloquear agentes de usuario conocidos por scraping
	if ($http_user_agent ~* "(curl|wget)") {
		return 403;
	}

	# Regla 3: Bloquear patrones de query sospechosos (ej. SQL injection basica)
	if ($query_string ~* "(union.*select|drop%20table)") {
		return 403;
	}

	# Ruta normal
	location / {
		return 200 "hello";
	}
}
		\end{lstlisting}
		\medskip \medskip

		Activar el sitio y cargar:
		\begin{lstlisting}
sudo ln -sf /etc/nginx/sites-available/app.conf /etc/nginx/sites-enabled/app.conf
sudo nginx -t
sudo systemctl reload nginx
		\end{lstlisting}

		\medskip \medskip

		\item Probar las reglas activadas
		\begin{lstlisting}
# Metodo bloqueado en /public
curl -i -X POST http://localhost/public

# Agente de usuario bloqueado
curl -i -A "curl" http://localhost/

# Query sospechosa bloqueada
curl -i "http://localhost/?q=union%20select%201"

# Ruta normal
curl -i http://localhost/
		\end{lstlisting}

		\medskip \medskip

		Esperado: `405` en `/public` con POST, `403` en agente bloqueado y query sospechosa, `200` en base.

		\medskip \medskip
		
		\item Desactivar temporalmente reglas y observar

		El siguiente código comenta las líneas `if(...) { return ...; }` en `app.conf`, recarga y repite pruebas.
		\begin{lstlisting}
sudo sed -i 's/^\s*if (.*return.*/# &/' /etc/nginx/sites-available/app.conf
sudo nginx -t
sudo systemctl reload nginx
		\end{lstlisting}

		\medskip \medskip

		Observa que las respuestas previamente bloqueadas ahora retornan `200` (o `hello`).		

		\end{enumerate}
		
		\bigskip \bigskip
		\item \noindent \textbf{Parte B: Certificados, TLS/HTTPS y OpenSSL}

		\medskip
		Prerequisitos (Linux):
		\begin{itemize}
		\item `openssl` instalado: `sudo apt install -y openssl`
		\end{itemize}

		\medskip \medskip

		\item Criptografía en el handshake TLS

		Flujo simplificado:
		\begin{itemize}
		\item Cliente inicia TLS; servidor presenta su certificado (clave pública, identidad).
		\medskip
		\item Cliente valida cadena y hostname; acuerdan secretos de sesión (Diffie-Hellman/ECDHE).
		\medskip
		\item Tráfico de datos usa cifrado simétrico con AEAD (AES-GCM/ChaCha20-Poly1305).
		\end{itemize}

		\medskip \medskip
		\item Inspeccionar un certificado de un sitio real

		Observar campos X.509 y la cadena de confianza.
		\medskip \medskip
		Comandos:
		\medskip
		\begin{lstlisting}
# Obtener y mostrar handshake y cadena
openssl s_client -connect example.com:443 -servername example.com -showcerts </dev/null

# Extraer el certificado leaf a un archivo (ejemplo)
openssl s_client -connect example.com:443 -servername example.com </dev/null \
  | awk 'BEGIN{c=0} /BEGIN CERT/{c=1} {if(c) print} /END CERT/{exit}' > leaf.pem

# Inspeccionar campos del certificado
openssl x509 -in leaf.pem -noout -text

# Ver solo SAN (hostnames validos)
openssl x509 -in leaf.pem -noout -ext subjectAltName
		\end{lstlisting}

		\medskip \medskip
		Observa: versión, `Subject`, `Issuer` (CA), periodo de validez, `Subject Alternative Name`, `Key Usage`, `Signature Algorithm`.		
		
		\end{itemize}
		
		\bigskip \bigskip
		\item Crear un certificado autosignado para pruebas locales
		
		Para simular un entorno con cadena no confiable y entender advertencias del cliente.
		\medskip \medskip

		Comandos:
		\medskip
		\begin{lstlisting}
openssl req -x509 -newkey rsa:2048 -keyout key.pem -out cert.pem -days 30 -nodes -subj "/CN=localhost"

# Servir con Nginx (bloque server adicional)
sudo mkdir -p /etc/nginx/ssl
sudo cp cert.pem key.pem /etc/nginx/ssl/
		\end{lstlisting}

		\medskip \medskip
		Configura `/etc/nginx/sites-available/tls.conf`:
		\medskip
		\begin{lstlisting}
server {
	listen 443 ssl;
	server_name localhost;
	ssl_certificate     /etc/nginx/ssl/cert.pem;
	ssl_certificate_key /etc/nginx/ssl/key.pem;
	location / { return 200 "tls ok"; }
}
		\end{lstlisting}

		\medskip \medskip
		Activar y probar:
		\begin{lstlisting}
sudo ln -sf /etc/nginx/sites-available/tls.conf /etc/nginx/sites-enabled/tls.conf
sudo nginx -t
sudo systemctl reload nginx

# Probando con verificacion (fallara por CA no confiable)
curl -vk https://localhost/
		\end{lstlisting}

		\medskip \medskip
		Observa: `curl -k` ignora verificación; útil solo para pruebas. Para confianza real, instala la CA en el almacén del sistema.		
		
		\bigskip\bigskip
		%------------------------------------------------------------%

	\end{enumerate}
	
	Conclusiones:
	\medskip \medskip
	\begin{itemize}
	\item Firewalls de capa de aplicación como Nginx permiten políticas expresivas y testeables.
	\medskip
	\item HTTPS combina asimetría (autenticación/intercambio de claves) con simetría (canal de datos), mediado por certificados y cadenas de confianza.
	\medskip
	\item La práctica con `openssl` y `curl` consolida diagnóstico y comprensión operacional.
	\end{itemize}

\end{document}
