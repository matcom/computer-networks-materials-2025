\documentclass[12pt]{amsart}

\addtolength{\hoffset}{-2.25cm}
\addtolength{\textwidth}{4.5cm}
\addtolength{\voffset}{-2.5cm}
\addtolength{\textheight}{5cm}
\setlength{\parskip}{0pt}
\setlength{\parindent}{15pt}

\usepackage{amsthm}
\usepackage{amsmath}
\usepackage{amssymb}
\usepackage[spanish]{babel}
\usepackage[colorlinks = true, linkcolor = black, citecolor = black, final]{hyperref}

\usepackage{listings}
\lstset{
	language=Python,
	backgroundcolor=\color{gray!25},
	basicstyle=\small\ttfamily,
	breaklines=true
}

\usepackage{graphicx}
\usepackage{multicol}
\usepackage{ marvosym }
\usepackage{wasysym}
\usepackage{tikz}
\usetikzlibrary{patterns}

\newcommand{\ds}{\displaystyle}
\newcommand{\horrule}[1]{\rule{\linewidth}{#1}}

\setlength{\parindent}{0in}

\pagestyle{empty}
\begin{document}
	\hrule
	\smallskip
	\begin{center}
		{\scshape {\large Redes de Computadoras} \\
			Curso 2025-2026} \\ \smallskip
		\textbf{Clase Práctica \# 7} \\
		{\small \textbf{Tema:} Protocolo TCP (Transmission Control Protocol)}
	\end{center}
	\vspace{-8px}
	\rule{\linewidth}{2pt}
	
	{\scshape Facultad de Matemática y Computación}  \hfill {\scshape Universidad de La Habana}
	
	\bigskip\bigskip
	
	
	
	\vspace{1cm}
	
	\begin{center}
		\textit{La única forma de hacer un gran trabajo es amar lo que haces.}
	\end{center}
	
	\begin{flushright}
		- Steve Jobs
	\end{flushright}
	
	\vspace{1cm}
	
	% Ejercicios
	\begin{enumerate}
		
		\thispagestyle{empty}
		
		%------------------------------------------------------------%
		\item \textbf{\textit{Introducción y Características}}
		
		\medskip
		TCP es un protocolo de la capa de transporte denido en RFC 793 que ofrece servicio orientado a conexión,
confiable y de fujo de bytes ordenado.
		\medskip \medskip
		
		\noindent \textbf{Características principales:}

		\begin{itemize}
			
		\item \textbf{Orientado a conexión}: requiere establecimiento y cierre ordenado

		\item \textbf{Fiabilidad garantizada:} ACK, retransmisiones, checksum

		\item \textbf{Control de flujo:} ventana deslizante para evitar saturación

		\item \textbf{Control de congestión:} algoritmos para evitar colapso de red

		\item \textbf{Entrega ordenada:} números de secuencia para reensamblaje

		\item \textbf{Comunicación full-duplex:} transmisión bidireccional simultánea

		\end{itemize}
		
		\medskip \medskip

		\begin{table}[ht]
		\centering
		\begin{tabular}{|l|l|l|}
		\hline
		\textbf{Aspecto} & \textbf{TCP} & \textbf{UDP} \\ 
		\hline
		Conexión & Orientado a conexión & Sin conexión \\ 
		\hline
		Fiabilidad & Garantizada & No garantizada \\ 
		\hline
		Orden & Ordenado & Puede llegar desordenado \\ 
		\hline
		Control de congestión & Sí & No \\ 
		\hline
		Cabecera & 20 bytes a 60 bytes & 8 bytes \\ 
		\hline
		Overhead & Alto & Mínimo \\
		\hline
		\end{tabular}
		\medskip
		\caption{Comparación entre TCP y UDP}
		\label{tab:tcp_udp_comparison}
		\end{table}

		\bigskip\bigskip
		%------------------------------------------------------------%

		%------------------------------------------------------------%
		\item \textbf{\textit{ Formato del Segmento TCP}}
		
		\medskip
		Cabecera TCP (mínimo 20 bytes, hasta 60 con opciones):
		
		\medskip \medskip

		\begin{table}[ht]
		\centering
		\begin{tabular}{|c|c|c|c|c|c|c|c|c|c|c|c|c|c|c|c|c|c|c|c|c|c|c|c|c|c|c|c|c|c|c|c|}
		\hline
		\multicolumn{8}{|c|}{0} & \multicolumn{8}{|c|}{1} & \multicolumn{8}{|c|}{2} & \multicolumn{8}{|c|}{3} \\
		\hline
		0 & 1 & 2 & 3 & 4 & 5 & 6 & 7 & 0 & 1 & 2 & 3 & 4 & 5 & 6 & 7 & 0 & 1 & 2 & 3 & 4 & 5 & 6 & 7 & 0 & 1 & 2 & 3 & 4 & 5 & 6 & 7 \\
		\hline
		\multicolumn{16}{|c|}{\textbf{Source Port}} & \multicolumn{16}{|c|}{\textbf{Destination Port}} \\
		\hline
		\multicolumn{32}{|c|}{\textbf{Sequence Number}} \\
		\hline
		\multicolumn{32}{|c|}{\textbf{Acknowledgment Number}} \\
		\hline
		\multicolumn{5}{|c|}{Data offset} & \multicolumn{6}{|c|}{Reserved} & \multicolumn{5}{|c|}{\begin{tabular}{c|c|c|c|c|c} U&A&P&R&S&F \\ R&C&S&S&Y&I \\ G&K&H&T&N&N \\\end{tabular}} & \multicolumn{16}{|c|}{Window}  \\
		\hline
		\multicolumn{16}{|c|}{\textbf{Checksum}} & \multicolumn{16}{|c|}{\textbf{Urgent pointer}} \\
		\hline
		\multicolumn{27}{|c|}{\textbf{Options(0-40 bytes)}} & \multicolumn{5}{|c|}{\textbf{Padding}} \\
		\hline
		\end{tabular}
		\caption{Cabecera (Header) de un paquete TCP}
		\label{tab:tcp_packet}
		\end{table}
		
		\medskip \medskip		

		\noindent \textbf{Campos Principales:}
		\begin{itemize}
		\item \textbf{Puertos origen/destino:} identican aplicaciones
		\item \textbf{Número de secuencia:} posición del primer byte en el ujo
		\item \textbf{Número de ACK:} próximo byte esperado
		\item \textbf{Longitud de cabecera:} en palabras de 32 bits (mínimo 5)
		\item \textbf{Flags de control:} URG, ACK, PSH, RST, SYN, FIN
		\item \textbf{Ventana:} tamaño de ventana de recepción
		\item \textbf{Checksum:} vericación de integridad
		\end{itemize}
		
		\bigskip\bigskip
		%------------------------------------------------------------%

		%------------------------------------------------------------%
		\item \textbf{\textit{Establecimiento y Cierre de Conexión}}
		
		\medskip
		\noindent \textbf{3-Way Handshake (Establecimiento)}
		\medskip
		\begin{verbatim}
Cliente         Servidor
SYN seq=x  -->
           <--  SYN seq=y, ACK=x+1
ACK y+1    -->
		\end{verbatim}

		\medskip \medskip
		\noindent \textbf{Estados TCP}
		\medskip
		\begin{itemize}
		\item \textbf{Cliente}: CLOSED $\rightarrow$ SYN\_SENT $\rightarrow$ ESTABLISHED
		\item \textbf{Servidor}: CLOSED $\rightarrow$ LISTEN $\rightarrow$ SYN\_RCVD $\rightarrow$ ESTABLISHED
		\end{itemize}
	
		\medskip \medskip
		\noindent \textbf{4-Way Handshake (Cierre)}
		\medskip
		\begin{verbatim}
Un extremo	     Otro extremo
FIN seq=x  -->
           <--  ACK x+1
           <--  FIN seq=y
ACK y+1    -->
		\end{verbatim}

		\noindent \textbf{TIME\_WAIT:} Espera de 2*MSL para evitar problemas con segmentos duplicados.
		
		\bigskip\bigskip
		%------------------------------------------------------------%

		%------------------------------------------------------------%
		\item \textbf{\textit{Mecanismos de Fiabilidad}}
		
		\medskip \medskip
		\noindent \textbf{Números de Secuencia y ACK}
		\medskip
		\begin{itemize}
		\item Cada byte transmitido tiene un número de secuencia
		\item ACK indica el próximo byte esperado (conrmación acumulativa)
		\end{itemize}

		\medskip \medskip		

		\noindent \textbf{Retransmisiones}
		\medskip
		\begin{itemize}
		\item \textbf{Timeout basado en RTT:} Estimación dinámica del Round-Trip Time
		\item \textbf{Fast Retransmit:} Retransmisión tras 3 ACK duplicados
		\end{itemize}
		
		\bigskip\bigskip
		%------------------------------------------------------------%
		
		%------------------------------------------------------------%
		\item \textbf{\textit{Control de Flujo: Ventana Deslizante}}
		
		\medskip
		El receptor anuncia cuántos bytes puede aceptar (campo Window)
		
 		\medskip \medskip
		\begin{itemize}
		\item \textbf{Ventana de congestión (cwnd):} Limitación por congestión de red
		\item \textbf{Ventana de recepción (rwnd):} Limitación por buffer del receptor
		\item \textbf{Ventana efectiva} = min(cwnd, rwnd)
		\end{itemize}
		
		\bigskip\bigskip
		%------------------------------------------------------------%
		
		%------------------------------------------------------------%
		\item \textbf{\textit{Control de Congestión}}
		
		\medskip
		\noindent \textbf{Algoritmos Principales}
		
 		\medskip \medskip

		\noindent \textbf{Slow Start}
		\begin{itemize}
		\item cwnd(Ventana de congestión) comienza en 1 MSS
		\item Duplica cwnd por cada RTT (incremento exponencial)
		\end{itemize}

		\medskip \medskip

		\noindent \textbf{Congestion Avoidance:}
		\begin{itemize}
		\item Incremento aditivo: cwnd += 1 MSS por RTT
		\end{itemize}
		
		\medskip \medskip

		\noindent \textbf{Recuperación:}
		\begin{itemize}
		\item \textbf{Timeout:} ssthresh = cwnd/2, cwnd = 1 MSS
		\item \textbf{Fast Recovery:} ssthresh = cwnd/2, cwnd = ssthresh + 3 MSS
		\end{itemize}
		
		
		\bigskip\bigskip
		%------------------------------------------------------------%

		%------------------------------------------------------------%
		\item \textbf{\textit{Herramientas de Diagnóstico}}
		
		\medskip
		\noindent Comando ss -tulpn
		
 		\medskip \medskip

		\begin{lstlisting}
# Mostrar conexiones TCP
ss -tulpn
# Ejemplos practicos
ss -tlnp | grep :80 # Ver quien escucha en puerto 80
ss -t state established # Conexiones establecidas
ss -t state time-wait # Conexiones en TIME_WAIT
		\end{lstlisting}

		\medskip \medskip
		\noindent \textbf{Interpretación:}
		\medskip
		\begin{itemize}
		\item \textbf{Recv-Q/Send-Q:} Datos en cola de recepción/envío
		\item \textbf{Estado LISTEN:} Servidores esperando conexiones
		\item \textbf{Estado ESTAB:} Conexiones establecidas
		\end{itemize}
		
		\bigskip\bigskip
		%------------------------------------------------------------%

		%------------------------------------------------------------%
		\item \textbf{\textit{Análisis con Wireshark}}
		
		\medskip
		\noindent Filtros útiles:
		
 		\medskip \medskip

		\begin{lstlisting}
tcp.port == 80 # Trafico en puerto 80
tcp.flags.syn == 1 # Segmentos SYN
tcp.analysis.retransmission # Retransmisiones
tcp.window\_size < 100 # Ventanas pequenas
		\end{lstlisting}

		\medskip \medskip
		\noindent Métricas importantes:
		\medskip
		\begin{itemize}
		\item \textbf{RTT:} Tiempo entre envío y ACK
		\item \textbf{Throughput:} Datos transferidos por tiempo
		\item \textbf{Pérdida de paquetes:} Retransmisiones
		\end{itemize}
		
		\bigskip\bigskip
		%------------------------------------------------------------%

		%------------------------------------------------------------%
		\item \textbf{\textit{Implementación de Servidores}}
		
		\medskip
		\noindent Servidor TCP Multithread
		\medskip \medskip
		\begin{lstlisting}
import socket
import threading
def handle_client(conn, addr):
	print(f"Conexion desde {addr}")
	with conn:
		while True:
			data = conn.recv(1024)
			if not data:
				break
			conn.sendall(data) # Echo server

def server():
	HOST, PORT = '0.0.0.0', 8080
	with socket.socket(socket.AF_INET, socket.SOCK_STREAM) as s:
		s.setsockopt(socket.SOL_SOCKET, socket.SO_REUSEADDR, 1)
		s.bind((HOST, PORT))
		s.listen(5)
		print(f"Servidor en {HOST}:{PORT}")

		while True:
			conn, addr = s.accept()
			thread = threading.Thread(target=handle_client, args=(conn,addr))
			thread.start()
		\end{lstlisting}

		\medskip \medskip

		\noindent \textbf{Cliente de prueba}
		\medskip \medskip
		\begin{lstlisting}
import socket

def cliente():
	HOST, PORT = '127.0.0.1', 8080
	with socket.socket(socket.AF_INET, socket.SOCK_STREAM) as s:
		s.connect((HOST, PORT))
		s.sendall(b"Hola servidor TCP")
		data = s.recv(1024)
		print(f"Recibido: {data.decode()}")
		\end{lstlisting}

		\bigskip \bigskip
		%------------------------------------------------------------%
		
		%------------------------------------------------------------%
		\item \textbf{\textit{Casos de Uso y Mejores Prácticas}}
		
		\medskip
		\noindent Aplicaciones típicas:
		\medskip \medskip
		\begin{itemize}
		\item Transferencia web (HTTP/HTTPS)
		\item Correo electrónico (SMTP, IMAP, POP3)
		\item Transferencia de archivos (FTP, SFTP)
		\item Terminal remoto (SSH)
		\item Bases de datos
		\end{itemize}
		
		\bigskip\bigskip
		%------------------------------------------------------------%

		%------------------------------------------------------------%
		\item \textbf{\textit{Preguntas}}
		
		\medskip
		\noindent Preguntas de Discusión:
		\medskip \medskip
		\begin{itemize}
		\item ¿Por qué TCP necesita un handshake de 3 vías en lugar de 2?
		\item ¿Qué problemas resuelve el estado TIME-WAIT?
		\item ¿Cómo afecta el RTT al throughput máximo?
		\item ¿Cuándo es preferible usar I/O multiplexado sobre hilos?
		\item ¿Qué ventajas tiene ACK selectivo (SACK) sobre ACK acumulativo?
		\end{itemize}

		\medskip \medskip
		\noindent Preguntas Rápidas:
		\medskip \medskip
		\begin{itemize}
		\item ¿Qué flag indica establecimiento de conexión? (SYN)
		\item ¿Qué campo controla el fujo? (Window)
		\item ¿Cuántos bytes tiene la cabecera TCP mínima? (20 bytes)
		\item ¿Qué indica un ACK con número 1001? (Se recibieron bytes 1-1000)
		\end{itemize}
		
		\bigskip\bigskip
		%------------------------------------------------------------%

	\end{enumerate}	

\end{document}
